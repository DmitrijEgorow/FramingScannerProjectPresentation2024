% !TEX encoding = UTF-8 Unicode
%\documentclass{beamer} %voce pode usar este modelo tambem
\documentclass{beamer}%[handout]{beamer}

% for XeLaTex use this code below
\usepackage{fontspec}
% \setmainfont[
%   Ligatures=TeX,
%   Extension=.otf,
%   BoldFont=cmunbx,
%   ItalicFont=cmunti,
%   BoldItalicFont=cmunbi,
% ]{cmunrm}
% \setsansfont[
%   Ligatures=TeX,
%   Extension=.otf,
%   BoldFont=cmunsx,
%   ItalicFont=cmunsi,
% ]{cmunss}


%\setmainfont{Roboto}
%\setsansfont[Scale=MatchUppercase]{Roboto Light}
%\setmonofont[Scale=MatchUppercase]{Hack}
\usepackage{xfrac,unicode-math}
%\setmathfont{Noto Serif}

% for LuaLatex use this code below for russian language
%\usepackage{fontspec}
%\setmainfont{CMU Serif}[Ligatures=TeX]
%\setsansfont{CMU Sans Serif}[Ligatures=TeX]

\usepackage{inputenc}
%\usepackage[utf8x]{inputenc}         % кодовая страница документа
\usepackage{indentfirst}   % русский стиль: отступ первого абзаца раздела
%\usepackage[utf8]{inputenc}
\usepackage[english]{babel}
\usepackage{graphicx,url, lastpage}
\usepackage{wrapfig}
\usepackage{ulem} %зачеркнуть


%%% Работа с таблицами
\usepackage{array,tabularx,tabulary,booktabs} % Дополнительная работа с таблицами
\usepackage{longtable}  % Длинные таблицы
\usepackage{multirow} % Слияние строк в таблице
\usepackage{makecell} % Перенос в таблице



\batchmode
% \usepackage{pgfpages}
% \pgfpagesuselayout{4 on 1}[letterpaper,landscape,border shrink=5mm]
\usepackage{amsmath,amssymb,enumerate,epsfig,bbm,calc,color,ifthen,capt-of}
%\usetheme{Berlin}
\usetheme{Berlin}
% skips the subsection headlines
\setbeamertemplate{headline}
{%
  \begin{beamercolorbox}[colsep=1.5pt]{upper separation line head}
  \end{beamercolorbox}
  \begin{beamercolorbox}{section in head/foot}
    \vskip2pt\insertnavigation{\paperwidth}\vskip2pt
  \end{beamercolorbox}%
  \begin{beamercolorbox}[colsep=1.5pt]{lower separation line head}
  \end{beamercolorbox}
}
% removes footline
\setbeamertemplate{footline}{}
% insert page number
\addtobeamertemplate{navigation symbols}{}{%
  \usebeamerfont{footline}%
  \usebeamercolor[fg]{footline}%
  \hspace{2em}%
  %\insertframenumber/\inserttotalframenumber
  \raisebox{1.3pt}[0pt][0pt]{\insertframenumber/\inserttotalframenumber}
}
\usepackage{enumitem}
%
%\setitemize{label=\usebeamerfont*{itemize item}%
%  \usebeamercolor[fg]{itemize item}
%  \usebeamertemplate{itemize item}}

\setlist[itemize]{%
  label=\usebeamerfont*{itemize item}%
  \usebeamercolor[fg]{itemize item}%
  \usebeamertemplate{itemize item}%
}%http://tex.stackexchange.com/a/24491
\setlist[enumerate,1]{%
  label=\protect\usebeamerfont{enumerate item}%
  \protect\usebeamercolor[fg]{enumerate item}%
  \insertenumlabel.%
}
% definition of new key for enumitem:
\makeatletter
\enitkv@key{enumitem}{overlay}{%
  \beamerdefaultoverlayspecification{#1}%
}
\makeatother
%\setbeamerfont{title}{shape=\itshape,family=\rmfamily}
%\setbeamerfont{alerted text}{shape=\itshape,family=\rmfamily}
%\setbeamerfont{palette primary}{shape=\itshape,family=\rmfamily}
%\setbeamerfont{palette secondary}{shape=\itshape,family=\rmfamily}
%\setbeamerfont{palette tertiary}{shape=\itshape,family=\rmfamily}
%\setbeamerfont{palette quaternary}{shape=\itshape,family=\rmfamily}
%\setbeamerfont{headline}{shape=\itshape,family=\rmfamily, size=\tiny}
%\setbeamerfont{headline}{size*={6pt}}

%\usecolortheme{senac}



\newcommand{\N}{\mathbb{N}}
\newcommand{\Z}{\mathbb{Z}}
\newcommand{\R}{\mathbb{R}}




% Multiple columns
\usepackage{multicol}



%listings for code

\usepackage{color} %% это для отображения цвета в коде
\usepackage{listings} %% собственно, это и есть пакет listings

\usepackage{caption}
\DeclareCaptionFont{white}{\color{black}} %% это сделает текст заголовка белым white
% black, blue, brown, cyan, darkgray, gray, green, lightgray, lime, magenta, olive, orange, pink, purple, red, teal, violet, white, yellow
%% код ниже нарисует серую рамочку вокруг заголовка кода.
\definecolor{light-gray}{gray}{0.9}
\DeclareCaptionFormat{listing}{\colorbox{light-gray}{\parbox{\textwidth}{#1#2#3}}}
\captionsetup[lstlisting]{format=listing,labelfont=white,textfont=white}


\lstset{ %
  language=Java,                 % выбор языка для подсветки (здесь это С)
  basicstyle=\small\sffamily, % размер и начертание шрифта для подсветки кода
  morekeywords={class},
  numbers=left,               % где поставить нумерацию строк (слева\справа)
  numberstyle=\tiny,           % размер шрифта для номеров строк
  stepnumber=1,                   % размер шага между двумя номерами строк
  numbersep=5pt,                % как далеко отстоят номера строк от подсвечиваемого кода
  backgroundcolor=\color{white}, % цвет фона подсветки - используем \usepackage{color}
  showspaces=false,            % показывать или нет пробелы специальными отступами
  showstringspaces=false,      % показывать или нет пробелы в строках
  showtabs=false,             % показывать или нет табуляцию в строках
  frame=single,              % рисовать рамку вокруг кода
  tabsize=2,                 % размер табуляции по умолчанию равен 2 пробелам
  captionpos=t,              % позиция заголовка вверху [t] или внизу [b]
  breaklines=true,           % автоматически переносить строки (да\нет)
  breakatwhitespace=false, % переносить строки только если есть пробел
  escapeinside={\%*}{*)}   % если нужно добавить комментарии в коде
}




% \textbf{Automated Testing System for Android Applications of Samsung Innovation Campus Learners}

% \textbf{Supervisor: Olga V. Maksimenkova, \\
% Associate Professor, \\ Faculty of Computer Science / School of Software Engineering}



%-------------------------Titulo/Autores/Orientador------------------------------------------------
\title[Developing an Android Application for Indoor Control Inspections of Construction Progress]{
  Developing an Android Application for Indoor Control Inspections of \\ Construction Progress}
%\date{Fuad Aleskerov,  Dmitry Egorov, Vyacheslav Yakuba \\ \ \\  Higher School of Economics \\ \ \\  May 23, 2022 }%\\ Moscow, 2022 }
\date{ Dmitry Egorov \\ \ \\   Supervisor \\ Aleksey D. Shadrikov \\ Senior Specialist \\ AI Track Curator at Samsung Innovation Campus \\ Samsung Electronics Rus Company  \\ \ \\
HSE University \\ 29 June 2024 }%\\ Moscow, 2022 }
\author[Dmitry Egorov \hspace{10em} \insertpagenumber \ /  \pageref{LastPage}]{        }

%-------------------------Logo na parte de baixo do slide------------------------------------------
%\pgfdeclareimage[height=0.7cm]{senac-logo}{senac-logo.pdf}
%\logo{\pgfuseimage{senac-logo}\hspace*{0.5cm}}


%----------------------------- Declare
%\pgfdeclareimage[height=2.7cm]{turnir}{togo7}


%-------------------------Este código faz o menuzinho bacana na parte superior do slide------------
% \AtBeginSection[]
% {
%   \begin{frame}<beamer>
%     \frametitle{Содержание}
%     \tableofcontents[currentsection]
%   \end{frame}
% }
% \beamerdefaultoverlayspecification{<+->}
% -----------------------------------------------------------------------------
\begin{document}
% -----------------------------------------------------------------------------

\newenvironment{changemargin}{%
\begin{list}{}{%
\setlength{\topsep}{0pt}%
\setlength{\leftmargin}{-0.4cm}%
\setlength{\rightmargin}{-0.4cm}%
\setlength{\listparindent}{\parindent}%
\setlength{\itemindent}{\parindent}%
\setlength{\parsep}{\parskip}%
}%
\item[]}{\end{list}}

%---Gerador de Sumário---------------------------------------------------------
\frame{\titlepage}
% \begin{frame}{Содержание}
%   \tableofcontents
%  \end{frame}
\setlist{nolistsep, itemsep=3pt,parsep=0pt,leftmargin=1pt}

%\logo{}

%-----------------------------------------------------------------------------
\section[System overview]{System overview}
\begin{frame}{Introduction}
%\footnotesize

Mobile devices → {\usebeamercolor[fg]{structure} On-device inference }



%\begin{figure}[h]
%  \begin{minipage}[h]{0.4\linewidth}
%    \hbox{\hspace{-0.1em}\includegraphics[width=\linewidth]{img/three_stages_scheme.png}}
%    %\caption{\footnotesize Integration into existing infrastructure}
%  \end{minipage}
%\end{figure}



\end{frame}
%------------------------------------------------------------------------------
%-----------------------------------------------------------------------------
%\begin{frame}{Thesis (Research) Statement}
\begin{frame}{Introduction}
\footnotesize

We propose customisable approaches to the automated testing system specifically for Android labs

We focus on Samsung Innovation Campus courses

{\usebeamercolor[fg]{structure} Main goal}: to develop software for educational purposes capable of evaluating students' Android labs based on devised test scenarios

% \vspace{1.5ex}

Objectives:
\begin{enumerate}
\item Investigating features of currently available OJ systems
\item Examining the problems of flakiness and students' fraud
\item Describing outer components of the system that will enable integrations with existing LMSs
\item Building the architecture of the system and defining inner modules that could be reused or adapted
\item Implementing Android labs with UI tests
\item Testing the software with real students
\item Analysing the results and highlighting feasible improvements
\end{enumerate}



\end{frame}
%---------------------------------------------------------------------------
%------------------------------------------------------------------------------
\begin{frame}{Online judges}
\begin{changemargin}
\footnotesize




\begin{itemize}
\item<1-> code quality checks
\item<1-> plagiarism detection
\item<1-> reports with meaningful feedback
\item<1-> stress-tested
\item<1-> topics annotation
\item<1-> scalability
\end{itemize}

%{\usebeamercolor[fg]{structure} quantitative (with mathematical models)}  / qualitative (expert-based)

% \begin{figure}[h]
%  \begin{minipage}[h]{0.5\linewidth}
%  \hbox{\hspace{-0.1em}\includegraphics[width=\linewidth]{lebanon.png}}
%  \caption{\footnotesize Wheat trade (subgraph), 2021}
%  \end{minipage}
%  \end{figure}


\end{changemargin}
\end{frame}
%------------------------------------------------------------------------------
%------------------------------------------------------------------------------
\begin{frame}{Comparison of OJs}
\begin{changemargin}
\footnotesize



\end{changemargin}
\end{frame}
%------------------------------------------------------------------------------
%-----------------------------------------------------------------------------
\begin{frame}{Integration into existing infrastructure}
\footnotesize


\begin{figure}[h]
\begin{minipage}[h]{0.9\linewidth}
\hbox{\hspace{-0.1em}\includegraphics[width=0.9\linewidth]{img/three_stages_scheme.png}}
%\caption{\footnotesize Integration into existing infrastructure}
\end{minipage}
\end{figure}



\end{frame}
%---------------------------------------------------------------------------
%------------------------------------------------------------------------------
\begin{frame}{}
\begin{changemargin}
\footnotesize

\begin{figure}[h]
\begin{minipage}[h]{0.91\linewidth}
\hbox{\hspace{-2em}\includegraphics[width=0.91\linewidth]{img/core_specific_modules.png}}
\end{minipage}
\end{figure}

\end{changemargin}
\end{frame}
%------------------------------------------------------------------------------
%------------------------------------------------------------------------------
\begin{frame}{Continuous Integration (CI)}
\begin{changemargin}
\footnotesize


\begin{itemize}
\item Version Control System (VCS): self-hosted gitea at https://sicampus.ru/gitea/
\item Build automation: our Gradle scripts to override files with instrumented tests
\item Testing frameworks (\textit{e.g.} JUnit, Espresso, UIAutomator, Kapsresso) to ensure code quality and identify bugs
\end{itemize}




\end{changemargin}
\end{frame}
%------------------------------------------------------------------------------
%------------------------------------------------------------------------------
\begin{frame}{Flakiness}
\begin{changemargin}
\footnotesize




\end{changemargin}
\end{frame}
%------------------------------------------------------------------------------
%------------------------------------------------------------------------------
\begin{frame}{Assessment}
\begin{changemargin}
\footnotesize


\begin{figure}[h]
\begin{minipage}[h]{0.9\linewidth}
\hbox{\hspace{-2em}\includegraphics[width=0.5\linewidth]{img/test_matrix.png}}
\end{minipage}
\end{figure}

We launch our tests on 3 different devices simultaneously and send the maximum score across all runs to the Android Bundle

Tablets and phones (including Samsung Flip models)

\end{changemargin}
\end{frame}
%------------------------------------------------------------------------------
%------------------------------------------------------------------------------
\begin{frame}{Generating task description}
\begin{changemargin}
\footnotesize

\begin{figure}[h]
\begin{minipage}[h]{0.9\linewidth}
\hbox{\hspace{-2em}\includegraphics[width=0.9\linewidth]{img/randomisation_discord}}
\end{minipage}
\end{figure}



\end{changemargin}
\end{frame}
%------------------------------------------------------------------------------
%------------------------------------------------------------------------------
\begin{frame}{}
\begin{changemargin}
\footnotesize

\begin{figure}[h]
\begin{minipage}[h]{0.92\linewidth}
\hbox{\hspace{-2em}\includegraphics[width=0.92\linewidth]{img/gradle_scheme.png}}
\end{minipage}
\end{figure}

\end{changemargin}
\end{frame}
%------------------------------------------------------------------------------
%------------------------------------------------------------------------------
\section[Results]{Results}
\begin{frame}{Results and Discussion}
\begin{changemargin}
\footnotesize
%\scriptsize

%\vspace{1.2ex}

Our comprehensive solution for evaluating students' programming assignments in the Android course

A modular and customisable approach to building an automated testing system tailored specifically for Android labs

Partitioning the unified test codebase into core and lab-specific components assigns different responsibilities to the platform team and educators

%Isolating key components for the segregation of duties within the core team and educators

Covering the topics of the Samsung Innovation Campus Android courses at http://itschool.innovationcampus.ru \vspace{1.2ex}

\begin{itemize}
\item<1-> core module for calculating students' marks and preparing the project (as git submodules)
\item<1-> approaches for giving feedback by adding extra messages to the test steps
\item<1-> performance checks, monkey testing, and feature toggles
\item<1-> monitoring devices and gathering analytics by Instabug and custom analytics server in Discord
\item<1-> pilot trial of the system on school and university students
\end{itemize}

Future directions: extend support for compose layouts, enhance collaboration to leverage students' collective expertise, and personalise learning trajectories based on student profiles.

%\vspace{1.10ex}26.01.2023: Joint Scientific Seminar of the Strategic Project <<National Centre for Scientific, Technological and Socio-Economic Foresight>>, presentation <<Network analysis under deep uncertainty: The problem of food security>>





\end{changemargin}
\end{frame}
%------------------------------------------------------------------------------
%------------------------------------------------------------------------------
\section[References]{References}
\begin{frame}{ }
\begin{changemargin}
\footnotesize

\begin{minipage}[t][0.8715\textheight]{\linewidth}
\begin{center}
Thank you! \vspace{1.5ex}
\end{center}

\scriptsize
\begin{enumerate}

\item<1->[[1\text{]}] 
\item<1->[[2\text{]}] 
\item<1->[[3\text{]}] 
\item<1->[[4\text{]}] 
\item<1->[[5\text{]}] 
\item<1->[[6\text{]}] 
\item<1->[[7\text{]}] 



\end{enumerate}


\vspace{3ex}% or \vfill
% Dmitry Egorov, tegorkrsk@gmail.com
\end{minipage}


%\vspace{20ex}

\end{changemargin}
\end{frame}
%------------------------------------------------------------------------------
%------------------------------------------------------------------------------

\begin{frame}{ }
\footnotesize

\begin{minipage}[t][0.8715\textheight]{\linewidth}
\begin{center}
Thank you! \vspace{1.5ex}
\end{center}

\scriptsize
\begin{enumerate}

\item<1->[[8\text{]}] S. H. Edwards and M. A. Pérez-Quiñones, “Experiences using test-driven development with an automated grader”, Journal of Computing Sciences in Colleges, vol. 22, no. 3, pp. 44–50, 2007.
\item<1->[[9\text{]}] K. Georgouli and P. Guerreiro, “Incorporating an automatic judge into blended learning programming activities”, in Advances in Web-Based Learning–ICWL 2010: 9th International Conference, Shanghai, China, December 8-10, 2010. Proceedings 9, Springer, 2010, pp. 81–90.
\item<1->[[10\text{]}]Z. Dong, A. Tiwari, X. L. Yu, and A. Roychoudhury, “Flaky test detection in android via event order exploration”, in Proceedings of the 29th ACM Joint Meeting on European Software Engineering Conference and Symposium on the Foundations of Software Engineering, 2021, pp. 367–378.
\item<1->[[11\text{]}] S. Thorve, C. Sreshtha, and N. Meng, “An empirical study of flaky tests in android apps”, in 2018 IEEE International Conference on Software Maintenance and Evolution (ICSME), IEEE, 2018, pp. 534–538.
\item<1->[[12\text{]}] S. Heckman and J. King, “Developing software engineering skills using real tools for automated grading”, in Proceedings of the 49th ACM technical symposium on computer science education, 2018, pp. 794–799.
\item<1->[[13\text{]}] R. P. Godinho, R. S. C. Espinosa, and C. Horgan, “Mobile software testing and evaluation on real devices in higher education: An irish open device lab case study”, Revista EducaOnline, vol. 15, no. 2, pp. 167–196, 2021.


\end{enumerate}


\vspace{3ex}% or \vfill
% Dmitry Egorov, tegorkrsk@gmail.com
\end{minipage}


%\vspace{20ex}

\end{frame}
%------------------------------------------------------------------------------
%------------------------------------------------------------------------------
\begin{frame}{Technologies}
\begin{changemargin}
\footnotesize


\begin{table}[h]
\caption{Frameworks that were used in the system}
\centering\begin{tabular}{rl}
\toprule
Library     & Description                           \\
\midrule
UIAutomator & Interacting directly with the device  \\
Kaspresso   & Launching UI tests                    \\
Kakao       & Launching UI tests                    \\
Diskord     & Connecting to Discord server via REST \\
JUnit       & Launching unit-tests                  \\
Marathon    & Device orchestration                  \\
\bottomrule
\end{tabular}
\label{table:libraries}
\end{table}


\end{changemargin}
\end{frame}
%------------------------------------------------------------------------------

\end{document}
%-----------------------------------------------Este comentario nunca aparecera